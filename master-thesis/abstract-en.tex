\documentclass[11pt,b5paper]{article}
% \usepackage{ucs}
% \usepackage{utf8x}
% \usepackage[utf8]{inputenc}
% \usepackage[10pt]{type1ec}          % use only 10pt fonts
% \usepackage{CJK}
% \usepackage[T1]{fontenc}
% \usepackage{pshan}
% \usepackage{hangul}
% \usepackage{hfont}
\usepackage{pslatex}
% \usepackage[margin=2in]{geometry}
\usepackage{setspace}
% \singlespacing{}
\onehalfspacing{}
% \doublespacing{}
\usepackage[margin=2.5cm]{geometry}
\pdfpagewidth 18cm
\pdfpageheight 25.5cm

% \newenvironment{Korean}{%
%   \CJKfamily{mj}}{}

\title{\huge{ABSTRACT}}
\date{}


\begin{document}
\maketitle{}
% \thispagestyle{empty}
\setcounter{page}{39}
In the vision of Computer Aided Diagnosis(CAD), to find and treat specific
illness, ROI (Region of Interest) analyze is an important methodology for
finding special characteristics. Particularly, for 2D (an image) and 3D
(a volume) datasets, many medical applications are designed to draw the
boundaries of a tumor for measuring its size.

In practical study of medical imaging, surface of a specific 3D contour might
be very huge. And it would be time-consuming task to propagate the contour and
triangulate it into regular mesh. This thesis presents a series of methods
from computing the contour tree to the simplification method. And then, taking
the advantage of both multi-core CPU and many-core GPU, boundary of interest
can be extracted with a hybrid accelerated method.
\vfill{}
\noindent\rule{\linewidth}{0.2mm}

\noindent\small{\textbf{Keywords}: \textit{Multi-threading, CUDA, Contour Tree, Isosurface, Contour Component}}



\end{document}
